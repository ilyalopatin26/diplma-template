%%% Работа с русским языком
\usepackage{cmap}			 % поиск в PDF
\usepackage{mathtext} 		 % русские буквы в формулах
\usepackage[T2A]{fontenc}	 % кодировка
\usepackage[utf8]{inputenc}	 % кодировка исходного текста
\usepackage[russian]{babel}	 % локализация и переносы

%%% Пакеты для работы с математикой
\usepackage{amsmath, amsfonts, amssymb, amsthm, mathtools, physics}
\usepackage{icomma}


%% Номера формул
\mathtoolsset{showonlyrefs=true} % Показывать номера только у тех формул, на которые есть \eqref{} в тексте.
%\usepackage{leqno}               % Немуреация формул слева
%% Перенос знаков в формулах (по Львовскому)
\newcommand*{\hm}[1]{#1\nobreak\discretionary{}{\hbox{$\mathsurround=0pt #1$}}{}}

%% Шрифты
\usepackage{euscript}	 % Шрифт Евклид
\usepackage{mathrsfs}    % Красивый матшрифт

%% Поля 
\usepackage[left=30mm,right=15mm,top=20mm,bottom=20mm,bindingoffset=0cm]{geometry}

%% Русские списки
\usepackage{enumitem}
\makeatletter
\AddEnumerateCounter{\asbuk}{\russian@alph}{щ}
\makeatother

%%% Работа с картинками
\usepackage{caption}
\captionsetup{justification=centering} % центрирование подписей к картинкам
\usepackage{graphicx}                  % Для вставки рисунков
\graphicspath{{images/}{images2/}}     % папки с картинками
\setlength\fboxsep{3pt}                % Отступ рамки \fbox{} от рисунка
\setlength\fboxrule{1pt}               % Толщина линий рамки \fbox{}
\usepackage{wrapfig}                   % Обтекание рисунков и таблиц текстом

%%% Работа с таблицами
\usepackage{array,tabularx,tabulary,booktabs} % Дополнительная работа с таблицами
\usepackage{longtable}                        % Длинные таблицы
\usepackage{multirow}                         % Слияние строк в таблице

%% Красная строка
\setlength{\parindent}{1.25cm}

%% Интервалы
%\linespread{1}
%\usepackage{multirow}

\usepackage{setspace}
%\полуторный интервал
\onehalfspacing

%% TikZ
\usepackage{tikz}
\usetikzlibrary{graphs,graphs.standard}

%% Верхний колонтитул
\usepackage{fancyhdr}
\pagestyle{fancy}


%% дополнения
\usepackage{float}   % Добавляет возможность работы с командой [H] которая улучшает расположение на странице
\usepackage{gensymb} % Красивые градусы
\usepackage{caption} % Пакет для подписей к рисункам, в частности, для работы caption*

% подключаем hyperref (для ссылок внутри  pdf)
\usepackage[unicode, pdftex]{hyperref}

%% быстрое оформление
\theoremstyle{plain} % стиль по умолчанию
\newtheorem{lemma}[section]{Лемма}

%% алгоритмы
\usepackage{algorithm}
\usepackage{algpseudocode}

\makeatletter
\algblock[ALGORITHMBLOCK]{BeginAlgorithm}{EndAlgorithm}
\algblock[BLOCK]{BeginBlock}{EndBlock}
\makeatother

% Нумерация блоков
\usepackage{caption}% http://ctan.org/pkg/caption
\captionsetup[ruled]{labelsep=period}
\makeatletter
\@addtoreset{algorithm}{chapter}% algorithm counter resets every chapter
\makeatother
\renewcommand{\thealgorithm}{\thechapter.\arabic{algorithm}}% Algorithm # is <chapter>.<algorithm>

 \renewcommand{\listalgorithmname}{Список алгоритмов}
 \floatname{algorithm}{Алгоритм}

  % Перевод команд псевдокода
  \algrenewcommand\algorithmicwhile{\textbf{До тех пока}}
  \algrenewcommand\algorithmicdo{\textbf{выполнять}}
  \algrenewcommand\algorithmicrepeat{\textbf{Повторять}}
  \algrenewcommand\algorithmicuntil{\textbf{Пока выполняется}}
  \algrenewcommand\algorithmicend{\textbf{Конец}}
  \algrenewcommand\algorithmicif{\textbf{Если}}
  \algrenewcommand\algorithmicelse{\textbf{иначе}}
  \algrenewcommand\algorithmicthen{\textbf{тогда}}
  \algrenewcommand\algorithmicfor{\textbf{Цикл}}
  \algrenewcommand\algorithmicforall{\textbf{Выполнить для всех}}
  \algrenewcommand\algorithmicfunction{\textbf{Функция}}
  \algrenewcommand\algorithmicprocedure{\textbf{Процедура}}
  \algrenewcommand\algorithmicloop{\textbf{Зациклить}}
  \algrenewcommand\algorithmicrequire{\textbf{Условия:}}
  \algrenewcommand\algorithmicensure{\textbf{Обеспечивающие условия:}}
  \algrenewcommand\algorithmicreturn{\textbf{Возвратить}}
  \algrenewtext{EndWhile}{\textbf{Конец цикла}}
  \algrenewtext{EndLoop}{\textbf{Конец зацикливания}}
  \algrenewtext{EndFor}{\textbf{Конец цикла}}
  \algrenewtext{EndFunction}{\textbf{Конец функции}}
  \algrenewtext{EndProcedure}{\textbf{Конец процедуры}}
  \algrenewtext{EndIf}{\textbf{Конец условия}}
  \algrenewtext{EndFor}{\textbf{Конец цикла}}
  \algrenewtext{BeginAlgorithm}{\textbf{Начало алгоритма}}
  \algrenewtext{EndAlgorithm}{\textbf{Конец алгоритма}}
  \algrenewtext{BeginBlock}{\textbf{Начало блока. }}
  \algrenewtext{EndBlock}{\textbf{Конец блока}}
  \algrenewtext{ElsIf}{\textbf{иначе если }}